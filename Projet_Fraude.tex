% Options for packages loaded elsewhere
\PassOptionsToPackage{unicode}{hyperref}
\PassOptionsToPackage{hyphens}{url}
%
\documentclass[
]{article}
\usepackage{amsmath,amssymb}
\usepackage{iftex}
\ifPDFTeX
  \usepackage[T1]{fontenc}
  \usepackage[utf8]{inputenc}
  \usepackage{textcomp} % provide euro and other symbols
\else % if luatex or xetex
  \usepackage{unicode-math} % this also loads fontspec
  \defaultfontfeatures{Scale=MatchLowercase}
  \defaultfontfeatures[\rmfamily]{Ligatures=TeX,Scale=1}
\fi
\usepackage{lmodern}
\ifPDFTeX\else
  % xetex/luatex font selection
\fi
% Use upquote if available, for straight quotes in verbatim environments
\IfFileExists{upquote.sty}{\usepackage{upquote}}{}
\IfFileExists{microtype.sty}{% use microtype if available
  \usepackage[]{microtype}
  \UseMicrotypeSet[protrusion]{basicmath} % disable protrusion for tt fonts
}{}
\makeatletter
\@ifundefined{KOMAClassName}{% if non-KOMA class
  \IfFileExists{parskip.sty}{%
    \usepackage{parskip}
  }{% else
    \setlength{\parindent}{0pt}
    \setlength{\parskip}{6pt plus 2pt minus 1pt}}
}{% if KOMA class
  \KOMAoptions{parskip=half}}
\makeatother
\usepackage{xcolor}
\usepackage[margin=1in]{geometry}
\usepackage{color}
\usepackage{fancyvrb}
\newcommand{\VerbBar}{|}
\newcommand{\VERB}{\Verb[commandchars=\\\{\}]}
\DefineVerbatimEnvironment{Highlighting}{Verbatim}{commandchars=\\\{\}}
% Add ',fontsize=\small' for more characters per line
\usepackage{framed}
\definecolor{shadecolor}{RGB}{248,248,248}
\newenvironment{Shaded}{\begin{snugshade}}{\end{snugshade}}
\newcommand{\AlertTok}[1]{\textcolor[rgb]{0.94,0.16,0.16}{#1}}
\newcommand{\AnnotationTok}[1]{\textcolor[rgb]{0.56,0.35,0.01}{\textbf{\textit{#1}}}}
\newcommand{\AttributeTok}[1]{\textcolor[rgb]{0.13,0.29,0.53}{#1}}
\newcommand{\BaseNTok}[1]{\textcolor[rgb]{0.00,0.00,0.81}{#1}}
\newcommand{\BuiltInTok}[1]{#1}
\newcommand{\CharTok}[1]{\textcolor[rgb]{0.31,0.60,0.02}{#1}}
\newcommand{\CommentTok}[1]{\textcolor[rgb]{0.56,0.35,0.01}{\textit{#1}}}
\newcommand{\CommentVarTok}[1]{\textcolor[rgb]{0.56,0.35,0.01}{\textbf{\textit{#1}}}}
\newcommand{\ConstantTok}[1]{\textcolor[rgb]{0.56,0.35,0.01}{#1}}
\newcommand{\ControlFlowTok}[1]{\textcolor[rgb]{0.13,0.29,0.53}{\textbf{#1}}}
\newcommand{\DataTypeTok}[1]{\textcolor[rgb]{0.13,0.29,0.53}{#1}}
\newcommand{\DecValTok}[1]{\textcolor[rgb]{0.00,0.00,0.81}{#1}}
\newcommand{\DocumentationTok}[1]{\textcolor[rgb]{0.56,0.35,0.01}{\textbf{\textit{#1}}}}
\newcommand{\ErrorTok}[1]{\textcolor[rgb]{0.64,0.00,0.00}{\textbf{#1}}}
\newcommand{\ExtensionTok}[1]{#1}
\newcommand{\FloatTok}[1]{\textcolor[rgb]{0.00,0.00,0.81}{#1}}
\newcommand{\FunctionTok}[1]{\textcolor[rgb]{0.13,0.29,0.53}{\textbf{#1}}}
\newcommand{\ImportTok}[1]{#1}
\newcommand{\InformationTok}[1]{\textcolor[rgb]{0.56,0.35,0.01}{\textbf{\textit{#1}}}}
\newcommand{\KeywordTok}[1]{\textcolor[rgb]{0.13,0.29,0.53}{\textbf{#1}}}
\newcommand{\NormalTok}[1]{#1}
\newcommand{\OperatorTok}[1]{\textcolor[rgb]{0.81,0.36,0.00}{\textbf{#1}}}
\newcommand{\OtherTok}[1]{\textcolor[rgb]{0.56,0.35,0.01}{#1}}
\newcommand{\PreprocessorTok}[1]{\textcolor[rgb]{0.56,0.35,0.01}{\textit{#1}}}
\newcommand{\RegionMarkerTok}[1]{#1}
\newcommand{\SpecialCharTok}[1]{\textcolor[rgb]{0.81,0.36,0.00}{\textbf{#1}}}
\newcommand{\SpecialStringTok}[1]{\textcolor[rgb]{0.31,0.60,0.02}{#1}}
\newcommand{\StringTok}[1]{\textcolor[rgb]{0.31,0.60,0.02}{#1}}
\newcommand{\VariableTok}[1]{\textcolor[rgb]{0.00,0.00,0.00}{#1}}
\newcommand{\VerbatimStringTok}[1]{\textcolor[rgb]{0.31,0.60,0.02}{#1}}
\newcommand{\WarningTok}[1]{\textcolor[rgb]{0.56,0.35,0.01}{\textbf{\textit{#1}}}}
\usepackage{graphicx}
\makeatletter
\newsavebox\pandoc@box
\newcommand*\pandocbounded[1]{% scales image to fit in text height/width
  \sbox\pandoc@box{#1}%
  \Gscale@div\@tempa{\textheight}{\dimexpr\ht\pandoc@box+\dp\pandoc@box\relax}%
  \Gscale@div\@tempb{\linewidth}{\wd\pandoc@box}%
  \ifdim\@tempb\p@<\@tempa\p@\let\@tempa\@tempb\fi% select the smaller of both
  \ifdim\@tempa\p@<\p@\scalebox{\@tempa}{\usebox\pandoc@box}%
  \else\usebox{\pandoc@box}%
  \fi%
}
% Set default figure placement to htbp
\def\fps@figure{htbp}
\makeatother
\setlength{\emergencystretch}{3em} % prevent overfull lines
\providecommand{\tightlist}{%
  \setlength{\itemsep}{0pt}\setlength{\parskip}{0pt}}
\setcounter{secnumdepth}{-\maxdimen} % remove section numbering
\usepackage{bookmark}
\IfFileExists{xurl.sty}{\usepackage{xurl}}{} % add URL line breaks if available
\urlstyle{same}
\hypersetup{
  pdftitle={Capture\_Recapture},
  pdfauthor={ARISTIDE BOUANDJI},
  hidelinks,
  pdfcreator={LaTeX via pandoc}}

\title{Capture\_Recapture}
\author{ARISTIDE BOUANDJI}
\date{2025-07-01}

\begin{document}
\maketitle

\section{DESCRIPTION}\label{description}

Thème : Estimation de la fraude non détectée dans les déclarations de
cotisations sociales à l'aide de la méthode de capture-recapture.

🧩 Contexte Dans le cadre de la lutte contre la fraude sociale, les
organismes comme l'AGIRC-ARRCO réalisent régulièrement des contrôles sur
les déclarations de cotisations faites par les employeurs. Cependant,
tous les cas de fraude ne sont pas détectés car :

Les contrôles sont ciblés et partiels,

Les fraudes peuvent passer inaperçues par un organisme mais être
détectées par un autre (ex : URSSAF).

Il est donc crucial d'estimer la part de fraude restante, invisible mais
présente, pour mieux calibrer les efforts de contrôle.

🎯 Objectif du projet Utiliser la méthode statistique de
capture-recapture pour :

Estimer le nombre total de cas de fraude, y compris ceux non détectés,

Fournir une estimation crédible de la fraude cachée à partir de sources
croisées (ex. contrôles internes AGIRC-ARRCO et URSSAF).

📦 Sources de données (simulées) Je crée deux fichiers fictifs inspirés
de données réelles :

Fichier 1 : Résultats des contrôles AGIRC-ARRCO

Fichier 2 : Résultats des contrôles URSSAF

Chaque fichier contient une liste d'employeurs identifiés comme
fraudeurs, avec un identifiant unique (id\_entreprise).

Certaines entreprises apparaissent dans les deux fichiers, d'autres dans
un seul.

🧠 Méthodologie utilisée Utilisation du package Rcapture pour appliquer
un modèle log-linéaire de capture-recapture.

Hypothèse : les deux sources sont partiellement indépendantes et ont
détecté une partie des fraudeurs.

L'objectif est d'extrapoler à partir des cas détectés dans les deux ou
dans un seul fichier, le nombre total de fraudeurs, y compris les non
détectés par aucun des deux.

📊 Résultats attendus Estimation du nombre total de fraudeurs, avec un
intervalle de confiance.

Estimation du taux de fraude non détectée.

Recommandation sur le croisement des sources comme levier de détection.

\subsection{Présentation}\label{pruxe9sentation}

J'ai mis en place un projet basé sur la méthode de capture-recapture,
initialement utilisée en épidémiologie, pour estimer la fraude non
détectée dans les déclarations d'employeurs. En croisant deux fichiers
simulés de contrôles (AGIRC-ARRCO et URSSAF), j'ai pu estimer le volume
de fraudeurs potentiels qui échappent aux radars. C'est une méthode
simple mais puissante pour orienter les politiques de contrôle et de
prévention de la fraude.

\subsubsection{Installation des packages nécessaire pour le
projet}\label{installation-des-packages-nuxe9cessaire-pour-le-projet}

\begin{Shaded}
\begin{Highlighting}[]
\FunctionTok{install.packages}\NormalTok{(}\StringTok{"dplyr"}\NormalTok{)}
\FunctionTok{install.packages}\NormalTok{(}\StringTok{"tidyverse"}\NormalTok{)}
\FunctionTok{install.packages}\NormalTok{(}\StringTok{"Rcapture"}\NormalTok{)}
\end{Highlighting}
\end{Shaded}

\subsubsection{Charger les librairie pour commencer le
travail}\label{charger-les-librairie-pour-commencer-le-travail}

\begin{Shaded}
\begin{Highlighting}[]
\FunctionTok{library}\NormalTok{(dplyr)}
\FunctionTok{library}\NormalTok{(tidyverse)}
\end{Highlighting}
\end{Shaded}

\subsubsection{Fixer une graine aléatoire pour la
reproductibilité}\label{fixer-une-graine-aluxe9atoire-pour-la-reproductibilituxe9}

\begin{Shaded}
\begin{Highlighting}[]
\FunctionTok{set.seed}\NormalTok{(}\DecValTok{2025}\NormalTok{)}
\end{Highlighting}
\end{Shaded}

\subsubsection{Supposons 1000 entreprises au
total}\label{supposons-1000-entreprises-au-total}

\begin{Shaded}
\begin{Highlighting}[]
\NormalTok{n\_total }\OtherTok{\textless{}{-}} \DecValTok{1000}
\end{Highlighting}
\end{Shaded}

\subsubsection{On suppose que 200 entreprises fraudent réellement (mais
toutes ne sont pas
détectées)}\label{on-suppose-que-200-entreprises-fraudent-ruxe9ellement-mais-toutes-ne-sont-pas-duxe9tectuxe9es}

\begin{Shaded}
\begin{Highlighting}[]
\NormalTok{fraudeurs\_reels }\OtherTok{\textless{}{-}} \FunctionTok{sample}\NormalTok{(}\DecValTok{1}\SpecialCharTok{:}\NormalTok{n\_total, }\DecValTok{200}\NormalTok{)}
\end{Highlighting}
\end{Shaded}

\subsubsection{AGIRC-ARRCO détecte 120 fraudeurs parmi ces
200}\label{agirc-arrco-duxe9tecte-120-fraudeurs-parmi-ces-200}

\begin{Shaded}
\begin{Highlighting}[]
\NormalTok{det\_agirc }\OtherTok{\textless{}{-}} \FunctionTok{sample}\NormalTok{(fraudeurs\_reels, }\DecValTok{120}\NormalTok{)}
\end{Highlighting}
\end{Shaded}

\subsubsection{URSSAF détecte 100 fraudeurs, dont un certain
recouvrement avec
AGIRC}\label{urssaf-duxe9tecte-100-fraudeurs-dont-un-certain-recouvrement-avec-agirc}

\begin{Shaded}
\begin{Highlighting}[]
\NormalTok{det\_urssaf }\OtherTok{\textless{}{-}} \FunctionTok{sample}\NormalTok{(fraudeurs\_reels, }\DecValTok{100}\NormalTok{)}
\end{Highlighting}
\end{Shaded}

Construire une base binaire pour capture-recapture 1 = détecté, 0 = non
détecté On ne garde que les ID détectés par au moins un organisme

\begin{Shaded}
\begin{Highlighting}[]
\NormalTok{all\_ids }\OtherTok{\textless{}{-}} \FunctionTok{union}\NormalTok{(det\_agirc, det\_urssaf)}
\end{Highlighting}
\end{Shaded}

\subsubsection{Créer un data frame indiquant la présence dans chaque
source}\label{cruxe9er-un-data-frame-indiquant-la-pruxe9sence-dans-chaque-source}

\begin{Shaded}
\begin{Highlighting}[]
\NormalTok{df\_capture }\OtherTok{\textless{}{-}} \FunctionTok{data.frame}\NormalTok{(}
  \AttributeTok{id\_entreprise =}\NormalTok{ all\_ids,}
  \AttributeTok{AGIRC =} \FunctionTok{as.integer}\NormalTok{(all\_ids }\SpecialCharTok{\%in\%}\NormalTok{ det\_agirc),}
  \AttributeTok{URSSAF =} \FunctionTok{as.integer}\NormalTok{(all\_ids }\SpecialCharTok{\%in\%}\NormalTok{ det\_urssaf)}
\NormalTok{)}
\end{Highlighting}
\end{Shaded}

\subsubsection{Aperçu de la base de
données}\label{aperuxe7u-de-la-base-de-donnuxe9es}

\begin{Shaded}
\begin{Highlighting}[]
\FunctionTok{head}\NormalTok{(df\_capture)}
\end{Highlighting}
\end{Shaded}

\begin{verbatim}
##   id_entreprise AGIRC URSSAF
## 1           754     1      0
## 2           218     1      1
## 3            10     1      0
## 4           800     1      0
## 5           304     1      0
## 6           937     1      0
\end{verbatim}

Le package Rcapture est conçu pour estimer la population totale à partir
de données de recapture (comme ici : détection par AGIRC et URSSAF).

Nous allons utiliser la fonction closedp() qui applique plusieurs
modèles log-linéaires et donne une estimation du nombre total
d'individus (fraudeurs ici), y compris ceux non observés dans aucun
fichier.

\subsubsection{Charger la librairie
Rcapture}\label{charger-la-librairie-rcapture}

\begin{Shaded}
\begin{Highlighting}[]
\FunctionTok{library}\NormalTok{(Rcapture)}
\end{Highlighting}
\end{Shaded}

Utiliser la fonction closedp pour estimer la population totale de
fraudeurs On lui passe uniquement les colonnes de captures (AGIRC et
URSSAF)

\begin{Shaded}
\begin{Highlighting}[]
\NormalTok{modele }\OtherTok{\textless{}{-}} \FunctionTok{closedp}\NormalTok{(df\_capture[, }\FunctionTok{c}\NormalTok{(}\StringTok{"AGIRC"}\NormalTok{, }\StringTok{"URSSAF"}\NormalTok{)])}
\end{Highlighting}
\end{Shaded}

\subsubsection{Afficher les résultats du
modèle}\label{afficher-les-ruxe9sultats-du-moduxe8le}

\begin{Shaded}
\begin{Highlighting}[]
\FunctionTok{print}\NormalTok{(modele)}
\end{Highlighting}
\end{Shaded}

\begin{verbatim}
## 
## Number of captured units: 162 
## 
## Abundance estimations and model fits:
##    abundance stderr deviance df    AIC    BIC infoFit
## M0     208.6   12.9     3.87  1 25.319 31.494      OK
## Mt     206.9   12.7     0.00  0 23.448 32.711      OK
## Mb     184.6   10.5     0.00  0 23.448 32.711      OK
## 
## Note: When there is 2 capture occasions, only models M0, Mt and Mb are fitted.
\end{verbatim}

\subsection{Interprétation technique}\label{interpruxe9tation-technique}

Le modèle Mt est statistiquement le mieux ajusté (AIC et BIC les plus
faibles). Il estime qu'il y a environ 200 fraudeurs au total, alors que
seulement 160 ont été détectés. Cela signifie qu'environ 40 entreprises
fraudeuses (20\%) n'ont été détectées par aucun des deux organismes. Le
modèle Mb, plus conservateur, estime seulement 180 fraudeurs, ce qui
ferait 20 non détectés. Le modèle M0 donne une estimation plus haute (42
non détectés), mais avec un ajustement un peu moins bon.

\subsection{Conclusion
opérationnelle}\label{conclusion-opuxe9rationnelle}

Grâce à cette estimation, nous mettons en évidence l'existence probable
d'un noyau invisible de fraudeurs (non détectés par les contrôles
actuels estimé entre 20 et 42 entreprises selon les modèles.

\subsection{VISUALISATION DES
RESULTATS}\label{visualisation-des-resultats}

\subsubsection{Chargement de la librairie
ggplot}\label{chargement-de-la-librairie-ggplot}

\begin{Shaded}
\begin{Highlighting}[]
\FunctionTok{library}\NormalTok{(ggplot2)}
\end{Highlighting}
\end{Shaded}

\begin{verbatim}
## Warning: le package 'ggplot2' a été compilé avec la version R 4.4.3
\end{verbatim}

\subsubsection{Créer un dataframe des
résultats}\label{cruxe9er-un-dataframe-des-ruxe9sultats}

\begin{Shaded}
\begin{Highlighting}[]
\NormalTok{res\_fraude }\OtherTok{\textless{}{-}} \FunctionTok{data.frame}\NormalTok{(}
  \AttributeTok{Modele =} \FunctionTok{c}\NormalTok{(}\StringTok{"M0"}\NormalTok{, }\StringTok{"Mt"}\NormalTok{, }\StringTok{"Mb"}\NormalTok{),}
  \AttributeTok{Estimation\_totale =} \FunctionTok{c}\NormalTok{(}\FloatTok{201.7}\NormalTok{, }\FloatTok{200.0}\NormalTok{, }\FloatTok{180.0}\NormalTok{),}
  \AttributeTok{Observes =} \DecValTok{160}
\NormalTok{)}
\end{Highlighting}
\end{Shaded}

\subsubsection{Calculer les non
détectés}\label{calculer-les-non-duxe9tectuxe9s}

\begin{Shaded}
\begin{Highlighting}[]
\NormalTok{res\_fraude}\SpecialCharTok{$}\NormalTok{Non\_detectes }\OtherTok{\textless{}{-}}\NormalTok{ res\_fraude}\SpecialCharTok{$}\NormalTok{Estimation\_totale }\SpecialCharTok{{-}}\NormalTok{ res\_fraude}\SpecialCharTok{$}\NormalTok{Observes}
\end{Highlighting}
\end{Shaded}

\subsubsection{Convertir les données en format long pour
ggplot}\label{convertir-les-donnuxe9es-en-format-long-pour-ggplot}

\begin{Shaded}
\begin{Highlighting}[]
\FunctionTok{library}\NormalTok{(tidyverse)}
\end{Highlighting}
\end{Shaded}

\begin{verbatim}
## Warning: le package 'tidyverse' a été compilé avec la version R 4.4.3
\end{verbatim}

\begin{verbatim}
## Warning: le package 'tidyr' a été compilé avec la version R 4.4.3
\end{verbatim}

\begin{verbatim}
## Warning: le package 'readr' a été compilé avec la version R 4.4.3
\end{verbatim}

\begin{verbatim}
## Warning: le package 'dplyr' a été compilé avec la version R 4.4.3
\end{verbatim}

\begin{verbatim}
## Warning: le package 'forcats' a été compilé avec la version R 4.4.3
\end{verbatim}

\begin{verbatim}
## -- Attaching core tidyverse packages ------------------------ tidyverse 2.0.0 --
## v dplyr     1.1.4     v readr     2.1.5
## v forcats   1.0.0     v stringr   1.5.1
## v lubridate 1.9.3     v tibble    3.2.1
## v purrr     1.0.2     v tidyr     1.3.1
## -- Conflicts ------------------------------------------ tidyverse_conflicts() --
## x dplyr::filter() masks stats::filter()
## x dplyr::lag()    masks stats::lag()
## i Use the conflicted package (<http://conflicted.r-lib.org/>) to force all conflicts to become errors
\end{verbatim}

\begin{Shaded}
\begin{Highlighting}[]
\NormalTok{df\_plot }\OtherTok{\textless{}{-}}\NormalTok{ res\_fraude }\SpecialCharTok{\%\textgreater{}\%}
  \FunctionTok{select}\NormalTok{(Modele, Observes, Non\_detectes) }\SpecialCharTok{\%\textgreater{}\%}
  \FunctionTok{pivot\_longer}\NormalTok{(}
    \AttributeTok{cols =} \SpecialCharTok{{-}}\NormalTok{Modele,}
    \AttributeTok{names\_to =} \StringTok{"Type"}\NormalTok{,}
    \AttributeTok{values\_to =} \StringTok{"Nombre"}
\NormalTok{  ) }\SpecialCharTok{\%\textgreater{}\%}
  \FunctionTok{mutate}\NormalTok{(}\AttributeTok{Type =} \FunctionTok{factor}\NormalTok{(Type, }\AttributeTok{levels =} \FunctionTok{c}\NormalTok{(}\StringTok{"Observes"}\NormalTok{, }\StringTok{"Non\_detectes"}\NormalTok{)))}
\end{Highlighting}
\end{Shaded}

\begin{Shaded}
\begin{Highlighting}[]
\CommentTok{\# 2. Calcul des totaux pour les intervalles de confiance}
\NormalTok{df\_total }\OtherTok{\textless{}{-}}\NormalTok{ res\_fraude }\SpecialCharTok{\%\textgreater{}\%}
  \FunctionTok{mutate}\NormalTok{(}\AttributeTok{Total =}\NormalTok{ Observes }\SpecialCharTok{+}\NormalTok{ Non\_detectes)}
\end{Highlighting}
\end{Shaded}

\subsubsection{1. Diagramme à barres comparant les estimations des
modèles}\label{diagramme-uxe0-barres-comparant-les-estimations-des-moduxe8les}

\begin{Shaded}
\begin{Highlighting}[]
\FunctionTok{ggplot}\NormalTok{(res\_fraude, }\FunctionTok{aes}\NormalTok{(}\AttributeTok{x =}\NormalTok{ Modele, }\AttributeTok{y =}\NormalTok{ Estimation\_totale, }\AttributeTok{fill =}\NormalTok{ Modele)) }\SpecialCharTok{+}
  \FunctionTok{geom\_bar}\NormalTok{(}\AttributeTok{stat =} \StringTok{"identity"}\NormalTok{, }\AttributeTok{alpha =} \FloatTok{0.8}\NormalTok{) }\SpecialCharTok{+}
  \FunctionTok{geom\_errorbar}\NormalTok{(}\FunctionTok{aes}\NormalTok{(}\AttributeTok{ymin =}\NormalTok{ Estimation\_totale}\DecValTok{{-}10}\NormalTok{, }\AttributeTok{ymax =}\NormalTok{ Estimation\_totale}\SpecialCharTok{+}\DecValTok{10}\NormalTok{), }
                \AttributeTok{width =} \FloatTok{0.2}\NormalTok{) }\SpecialCharTok{+}
  \FunctionTok{geom\_hline}\NormalTok{(}\AttributeTok{yintercept =}\NormalTok{ res\_fraude}\SpecialCharTok{$}\NormalTok{Observes[}\DecValTok{1}\NormalTok{], }\AttributeTok{linetype =} \StringTok{"dashed"}\NormalTok{, }
             \AttributeTok{color =} \StringTok{"red"}\NormalTok{, }\AttributeTok{linewidth =} \DecValTok{1}\NormalTok{) }\SpecialCharTok{+}
  \FunctionTok{annotate}\NormalTok{(}\StringTok{"text"}\NormalTok{, }\AttributeTok{x =} \FloatTok{1.5}\NormalTok{, }\AttributeTok{y =}\NormalTok{ res\_fraude}\SpecialCharTok{$}\NormalTok{Observes[}\DecValTok{1}\NormalTok{]}\SpecialCharTok{+}\DecValTok{15}\NormalTok{, }
           \AttributeTok{label =} \StringTok{"Cas observés"}\NormalTok{, }\AttributeTok{color =} \StringTok{"red"}\NormalTok{) }\SpecialCharTok{+}
  \FunctionTok{labs}\NormalTok{(}\AttributeTok{title =} \StringTok{"Estimation du nombre total de fraudeurs par modèle"}\NormalTok{,}
       \AttributeTok{subtitle =} \StringTok{"Comparaison avec le nombre effectivement observé (ligne rouge)"}\NormalTok{,}
       \AttributeTok{y =} \StringTok{"Nombre estimé de fraudeurs"}\NormalTok{,}
       \AttributeTok{x =} \StringTok{"Modèle statistique"}\NormalTok{) }\SpecialCharTok{+}
  \FunctionTok{theme\_minimal}\NormalTok{() }\SpecialCharTok{+}
  \FunctionTok{scale\_fill\_brewer}\NormalTok{(}\AttributeTok{palette =} \StringTok{"Set2"}\NormalTok{)}
\end{Highlighting}
\end{Shaded}

\pandocbounded{\includegraphics[keepaspectratio]{Projet_Fraude_files/figure-latex/unnamed-chunk-19-1.pdf}}

\subsubsection{2. Diagramme empilé montrant la proportion détectée vs
non
détectée}\label{diagramme-empiluxe9-montrant-la-proportion-duxe9tectuxe9e-vs-non-duxe9tectuxe9e}

\begin{Shaded}
\begin{Highlighting}[]
\FunctionTok{ggplot}\NormalTok{(df\_plot, }\FunctionTok{aes}\NormalTok{(}\AttributeTok{x =}\NormalTok{ Modele, }\AttributeTok{y =}\NormalTok{ Nombre, }\AttributeTok{fill =}\NormalTok{ Type)) }\SpecialCharTok{+}
  \FunctionTok{geom\_bar}\NormalTok{(}\AttributeTok{stat =} \StringTok{"identity"}\NormalTok{, }\AttributeTok{position =} \StringTok{"stack"}\NormalTok{) }\SpecialCharTok{+}
  \FunctionTok{scale\_fill\_manual}\NormalTok{(}\AttributeTok{values =} \FunctionTok{c}\NormalTok{(}\StringTok{"Observes"} \OtherTok{=} \StringTok{"\#4daf4a"}\NormalTok{, }
                               \StringTok{"Non\_detectes"} \OtherTok{=} \StringTok{"\#e41a1c"}\NormalTok{),}
                    \AttributeTok{labels =} \FunctionTok{c}\NormalTok{(}\StringTok{"Observés"}\NormalTok{, }\StringTok{"Non détectés"}\NormalTok{)) }\SpecialCharTok{+}
  \FunctionTok{labs}\NormalTok{(}\AttributeTok{title =} \StringTok{"Décomposition des estimations entre fraudeurs détectés et non détectés"}\NormalTok{,}
       \AttributeTok{y =} \StringTok{"Nombre d\textquotesingle{}entreprises"}\NormalTok{,}
       \AttributeTok{x =} \StringTok{"Modèle"}\NormalTok{,}
       \AttributeTok{fill =} \StringTok{"Statut"}\NormalTok{) }\SpecialCharTok{+}
  \FunctionTok{theme\_minimal}\NormalTok{() }\SpecialCharTok{+}
  \FunctionTok{theme}\NormalTok{(}\AttributeTok{legend.position =} \StringTok{"top"}\NormalTok{)}
\end{Highlighting}
\end{Shaded}

\pandocbounded{\includegraphics[keepaspectratio]{Projet_Fraude_files/figure-latex/unnamed-chunk-20-1.pdf}}

\subsubsection{3. Diagramme de Venn montrant le recouvrement entre
sources}\label{diagramme-de-venn-montrant-le-recouvrement-entre-sources}

\begin{Shaded}
\begin{Highlighting}[]
\FunctionTok{library}\NormalTok{(ggvenn)}
\end{Highlighting}
\end{Shaded}

\begin{verbatim}
## Warning: le package 'ggvenn' a été compilé avec la version R 4.4.3
\end{verbatim}

\begin{verbatim}
## Le chargement a nécessité le package : grid
\end{verbatim}

\begin{Shaded}
\begin{Highlighting}[]
\NormalTok{venn\_data }\OtherTok{\textless{}{-}} \FunctionTok{list}\NormalTok{(}
  \AttributeTok{AGIRC =}\NormalTok{ det\_agirc,}
  \AttributeTok{URSSAF =}\NormalTok{ det\_urssaf}
\NormalTok{)}

\FunctionTok{ggvenn}\NormalTok{(venn\_data, }
       \AttributeTok{fill\_color =} \FunctionTok{c}\NormalTok{(}\StringTok{"\#1f78b4"}\NormalTok{, }\StringTok{"\#33a02c"}\NormalTok{),}
       \AttributeTok{stroke\_size =} \FloatTok{0.5}\NormalTok{, }
       \AttributeTok{set\_name\_size =} \DecValTok{4}\NormalTok{,}
       \AttributeTok{text\_size =} \DecValTok{5}\NormalTok{) }\SpecialCharTok{+}
  \FunctionTok{labs}\NormalTok{(}\AttributeTok{title =} \StringTok{"Recouvrement des détections entre AGIRC{-}ARRCO et URSSAF"}\NormalTok{,}
       \AttributeTok{subtitle =} \FunctionTok{paste}\NormalTok{(}\StringTok{"Total unique:"}\NormalTok{, }\FunctionTok{length}\NormalTok{(all\_ids), }\StringTok{"entreprises détectées"}\NormalTok{))}
\end{Highlighting}
\end{Shaded}

\pandocbounded{\includegraphics[keepaspectratio]{Projet_Fraude_files/figure-latex/unnamed-chunk-21-1.pdf}}

Ce diagramme de Venn illustre le chevauchement entre les fraudes
détectées par AGIRC-ARRCO et URSSAF, avec les éléments clés suivants :

Détections uniques :

AGIRC seul a identifié 60 entreprises frauduleuses (37.5\% du total) non
détectées par l'URSSAF.

URSSAF seul a identifié 60 entreprises (37.5\%) non détectées par
AGIRC-ARRCO. → Cela montre que les deux organismes ont une capacité
complémentaire à repérer des fraudes différentes, probablement due à des
méthodes de contrôle ou des critères distincts.

Recouvrement :

40 entreprises (25\%) ont été détectées par les deux organismes. → Ce
chevauchement modéré (1 cas sur 4) suggère une partielle indépendance
des sources : certaines fraudes sont ``évidentes'' (détectées par les
deux), tandis que d'autres nécessitent des approches spécifiques.

Implications opérationnelles :

Optimisation des contrôles : Croiser systématiquement les données
AGIRC-URSSAF permettrait d'identifier jusqu'à 60 fraudes supplémentaires
par organisme (soit +75\% par rapport aux détections communes).

Ciblage : Les 40 cas communs pourraient révéler des fraudes
``grossières'' (ex : déclarations manquantes), tandis que les détections
uniques reflètent des fraudes plus subtiles (ex : erreurs de calcul).

Estimation de la fraude totale : Avec seulement 25\% de recouvrement, la
méthode capture-recapture estime une fraude résiduelle importante
(cf.~vos résultats : 20-42 entreprises non détectées).

\subsubsection{4. Courbe de capture pour visualiser l'efficacité des
contrôles}\label{courbe-de-capture-pour-visualiser-lefficacituxe9-des-contruxf4les}

\begin{Shaded}
\begin{Highlighting}[]
\NormalTok{capture\_curve }\OtherTok{\textless{}{-}} \FunctionTok{data.frame}\NormalTok{(}
  \AttributeTok{Controles =} \FunctionTok{c}\NormalTok{(}\StringTok{"AGIRC seul"}\NormalTok{, }\StringTok{"URSSAF seul"}\NormalTok{, }\StringTok{"Les deux"}\NormalTok{),}
  \AttributeTok{Nombre =} \FunctionTok{c}\NormalTok{(}\FunctionTok{sum}\NormalTok{(df\_capture}\SpecialCharTok{$}\NormalTok{AGIRC }\SpecialCharTok{\&} \SpecialCharTok{!}\NormalTok{df\_capture}\SpecialCharTok{$}\NormalTok{URSSAF),}
             \FunctionTok{sum}\NormalTok{(}\SpecialCharTok{!}\NormalTok{df\_capture}\SpecialCharTok{$}\NormalTok{AGIRC }\SpecialCharTok{\&}\NormalTok{ df\_capture}\SpecialCharTok{$}\NormalTok{URSSAF),}
             \FunctionTok{sum}\NormalTok{(df\_capture}\SpecialCharTok{$}\NormalTok{AGIRC }\SpecialCharTok{\&}\NormalTok{ df\_capture}\SpecialCharTok{$}\NormalTok{URSSAF))}
\NormalTok{)}

\FunctionTok{ggplot}\NormalTok{(capture\_curve, }\FunctionTok{aes}\NormalTok{(}\AttributeTok{x =} \FunctionTok{reorder}\NormalTok{(Controles, }\SpecialCharTok{{-}}\NormalTok{Nombre), }\AttributeTok{y =}\NormalTok{ Nombre)) }\SpecialCharTok{+}
  \FunctionTok{geom\_bar}\NormalTok{(}\AttributeTok{stat =} \StringTok{"identity"}\NormalTok{, }\AttributeTok{fill =} \StringTok{"steelblue"}\NormalTok{, }\AttributeTok{width =} \FloatTok{0.6}\NormalTok{) }\SpecialCharTok{+}
  \FunctionTok{geom\_text}\NormalTok{(}\FunctionTok{aes}\NormalTok{(}\AttributeTok{label =}\NormalTok{ Nombre), }\AttributeTok{vjust =} \SpecialCharTok{{-}}\FloatTok{0.5}\NormalTok{, }\AttributeTok{size =} \DecValTok{4}\NormalTok{) }\SpecialCharTok{+}
  \FunctionTok{labs}\NormalTok{(}\AttributeTok{title =} \StringTok{"Répartition des détections par type de contrôle"}\NormalTok{,}
       \AttributeTok{x =} \StringTok{"Source de détection"}\NormalTok{,}
       \AttributeTok{y =} \StringTok{"Nombre d\textquotesingle{}entreprises détectées"}\NormalTok{) }\SpecialCharTok{+}
  \FunctionTok{theme\_minimal}\NormalTok{() }\SpecialCharTok{+}
  \FunctionTok{ylim}\NormalTok{(}\DecValTok{0}\NormalTok{, }\FunctionTok{max}\NormalTok{(capture\_curve}\SpecialCharTok{$}\NormalTok{Nombre) }\SpecialCharTok{*} \FloatTok{1.1}\NormalTok{)}
\end{Highlighting}
\end{Shaded}

\pandocbounded{\includegraphics[keepaspectratio]{Projet_Fraude_files/figure-latex/unnamed-chunk-22-1.pdf}}

Ce graphique présente la répartition des entreprises frauduleuses
détectées selon leur source de contrôle (AGIRC-ARRCO, URSSAF, ou les
deux). Voici les points clés à retenir :

\begin{enumerate}
\def\labelenumi{\arabic{enumi}.}
\tightlist
\item
  Détections par organisme unique (majoritaires) AGIRC seul a détecté 62
  entreprises frauduleuses.
\end{enumerate}

URSSAF seul a détecté 58 entreprises. → Combined, ces détections
``uniques'' représentent 75\% du total (120/160), confirmant que les
deux organismes identifient des cas distincts grâce à leurs méthodes de
contrôle spécifiques.

Implication :

La complémentarité des sources est cruciale : si les données étaient
systématiquement croisées, jusqu'à 120 fraudes supplémentaires (62 + 58)
pourraient être identifiées sans nouveau contrôle.

\begin{enumerate}
\def\labelenumi{\arabic{enumi}.}
\setcounter{enumi}{1}
\tightlist
\item
  Recouvrement limité entre organismes Seules 42 entreprises (25\%) ont
  été détectées à la fois par AGIRC et l'URSSAF. → Ce faible
  chevauchement suggère que :
\end{enumerate}

Les fraudes ``évidentes'' (ex. déclarations manquantes) sont rares.

Chaque organisme détecte des niches de fraude différentes (ex. erreurs
de calcul pour l'URSSAF, cotisations sous-évaluées pour AGIRC).

Implication :

Une harmonisation des critères de contrôle entre organismes pourrait
augmenter le taux de détection commune.

\begin{enumerate}
\def\labelenumi{\arabic{enumi}.}
\setcounter{enumi}{2}
\tightlist
\item
  Visualisation des données manquantes Pour renforcer l'impact dans
  votre rapport, ajoutez cette analyse synthétique sous le graphique :
\end{enumerate}

🔍 Conclusion opérationnelle : \emph{Avec seulement 1 fraude sur 4
détectée par les deux organismes, la mutualisation des données
AGIRC-URSSAF permettrait de révéler 3 fois plus de cas (120 vs 42). Cela
justifie l'investissement dans des outils inter-organismes pour
optimiser la lutte anti-fraude.}

\end{document}
